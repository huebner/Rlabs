\documentclass{article}\usepackage[]{graphicx}\usepackage[]{color}
%% maxwidth is the original width if it is less than linewidth
%% otherwise use linewidth (to make sure the graphics do not exceed the margin)
\makeatletter
\def\maxwidth{ %
  \ifdim\Gin@nat@width>\linewidth
    \linewidth
  \else
    \Gin@nat@width
  \fi
}
\makeatother

\definecolor{fgcolor}{rgb}{0.345, 0.345, 0.345}
\newcommand{\hlnum}[1]{\textcolor[rgb]{0.686,0.059,0.569}{#1}}%
\newcommand{\hlstr}[1]{\textcolor[rgb]{0.192,0.494,0.8}{#1}}%
\newcommand{\hlcom}[1]{\textcolor[rgb]{0.678,0.584,0.686}{\textit{#1}}}%
\newcommand{\hlopt}[1]{\textcolor[rgb]{0,0,0}{#1}}%
\newcommand{\hlstd}[1]{\textcolor[rgb]{0.345,0.345,0.345}{#1}}%
\newcommand{\hlkwa}[1]{\textcolor[rgb]{0.161,0.373,0.58}{\textbf{#1}}}%
\newcommand{\hlkwb}[1]{\textcolor[rgb]{0.69,0.353,0.396}{#1}}%
\newcommand{\hlkwc}[1]{\textcolor[rgb]{0.333,0.667,0.333}{#1}}%
\newcommand{\hlkwd}[1]{\textcolor[rgb]{0.737,0.353,0.396}{\textbf{#1}}}%

\usepackage{framed}
\makeatletter
\newenvironment{kframe}{%
 \def\at@end@of@kframe{}%
 \ifinner\ifhmode%
  \def\at@end@of@kframe{\end{minipage}}%
  \begin{minipage}{\columnwidth}%
 \fi\fi%
 \def\FrameCommand##1{\hskip\@totalleftmargin \hskip-\fboxsep
 \colorbox{shadecolor}{##1}\hskip-\fboxsep
     % There is no \\@totalrightmargin, so:
     \hskip-\linewidth \hskip-\@totalleftmargin \hskip\columnwidth}%
 \MakeFramed {\advance\hsize-\width
   \@totalleftmargin\z@ \linewidth\hsize
   \@setminipage}}%
 {\par\unskip\endMakeFramed%
 \at@end@of@kframe}
\makeatother

\definecolor{shadecolor}{rgb}{.97, .97, .97}
\definecolor{messagecolor}{rgb}{0, 0, 0}
\definecolor{warningcolor}{rgb}{1, 0, 1}
\definecolor{errorcolor}{rgb}{1, 0, 0}
\newenvironment{knitrout}{}{} % an empty environment to be redefined in TeX

\usepackage{alltt}
\usepackage[sc]{mathpazo}
\renewcommand{\sfdefault}{lmss}
\renewcommand{\ttdefault}{lmtt}
\usepackage[T1]{fontenc}
\usepackage{geometry}
\geometry{verbose,tmargin=2.5cm,bmargin=2.5cm,lmargin=2.5cm,rmargin=2.5cm}
\setcounter{secnumdepth}{2}
\setcounter{tocdepth}{2}
\usepackage[unicode=true,pdfusetitle,
 bookmarks=true,bookmarksnumbered=true,bookmarksopen=true,bookmarksopenlevel=2,
 breaklinks=false,pdfborder={0 0 1},backref=false,colorlinks=false]
 {hyperref}
\hypersetup{
 pdfstartview={XYZ null null 1}}
\usepackage{breakurl}

\makeatletter
%%%%%%%%%%%%%%%%%%%%%%%%%%%%%% User specified LaTeX commands.
\renewcommand{\textfraction}{0.05}
\renewcommand{\topfraction}{0.8}
\renewcommand{\bottomfraction}{0.8}
\renewcommand{\floatpagefraction}{0.75}

\makeatother
\IfFileExists{upquote.sty}{\usepackage{upquote}}{}

\begin{document}









The results below are generated from an R script.

\begin{knitrout}
\definecolor{shadecolor}{rgb}{0.969, 0.969, 0.969}\color{fgcolor}\begin{kframe}
\begin{alltt}
\hlcom{################### Rcode for inference for means (t-interval or t-test)}

\hlcom{# The production of a nationally marketed detergent results in certain workers receiving}
\hlcom{# prolonged exposure to the bacillus subtilis enzyme. Nineteen workers were tested to}
\hlcom{# determine the effects of these exposures on various respiratory functions.  The airflow}
\hlcom{# rate, FEV1, is the ratio of a person’s forced expiratory volume to the vital capacity, VC}
\hlcom{# (max. volume of air a person can exhale after taking a deep breath).  If the enzyme has}
\hlcom{# an effect, it will be to reduce the FEV1/VC ratio. The norm is 0.80 in persons with no}
\hlcom{# lung dysfunction.}

\hlstd{ratio} \hlkwb{<-} \hlkwd{c}\hlstd{(}\hlnum{0.61}\hlstd{,} \hlnum{0.7}\hlstd{,} \hlnum{0.63}\hlstd{,} \hlnum{0.76}\hlstd{,} \hlnum{0.67}\hlstd{,} \hlnum{0.72}\hlstd{,} \hlnum{0.64}\hlstd{,} \hlnum{0.82}\hlstd{,} \hlnum{0.88}\hlstd{,} \hlnum{0.82}\hlstd{,} \hlnum{0.78}\hlstd{,} \hlnum{0.84}\hlstd{,} \hlnum{0.83}\hlstd{,} \hlnum{0.82}\hlstd{,}
    \hlnum{0.74}\hlstd{,} \hlnum{0.85}\hlstd{,} \hlnum{0.73}\hlstd{,} \hlnum{0.85}\hlstd{,} \hlnum{0.87}\hlstd{)}

\hlcom{# check summary statistics}
\hlkwd{summary}\hlstd{(ratio)}
\end{alltt}
\begin{verbatim}
##    Min. 1st Qu.  Median    Mean 3rd Qu.    Max. 
##   0.610   0.710   0.780   0.766   0.835   0.880
\end{verbatim}
\begin{alltt}
\hlkwd{sd}\hlstd{(ratio)}
\end{alltt}
\begin{verbatim}
## [1] 0.08591
\end{verbatim}
\begin{alltt}
\hlcom{# are the data symmetric or approximately normsl?}
\hlkwd{qqnorm}\hlstd{(ratio)}
\hlkwd{qqline}\hlstd{(ratio)}
\end{alltt}
\end{kframe}

{\centering \includegraphics[width=.6\linewidth]{figure/Rcode-onesampleT-Rnwunnamed-chunk-1} 

}


\begin{kframe}\begin{alltt}
\hlcom{# t interval and t test is the same function use ?t.test to check what options are}
\hlcom{# available A 90% confidence interval can be obtained with}
\hlkwd{t.test}\hlstd{(ratio,} \hlkwc{mu} \hlstd{=} \hlnum{0.8}\hlstd{,} \hlkwc{conf.level} \hlstd{=} \hlnum{0.9}\hlstd{)}
\end{alltt}
\begin{verbatim}
## 
## 	One Sample t-test
## 
## data:  ratio
## t = -1.709, df = 18, p-value = 0.1046
## alternative hypothesis: true mean is not equal to 0.8
## 90 percent confidence interval:
##  0.7321 0.8005
## sample estimates:
## mean of x 
##    0.7663
\end{verbatim}
\begin{alltt}
\hlcom{# save the output:}
\hlstd{temp} \hlkwb{<-} \hlkwd{t.test}\hlstd{(ratio,} \hlkwc{mu} \hlstd{=} \hlnum{0.8}\hlstd{,} \hlkwc{conf.level} \hlstd{=} \hlnum{0.9}\hlstd{)}

\hlcom{# check what you can read from the output}
\hlkwd{names}\hlstd{(temp)}
\end{alltt}
\begin{verbatim}
## [1] "statistic"   "parameter"   "p.value"     "conf.int"    "estimate"    "null.value" 
## [7] "alternative" "method"      "data.name"
\end{verbatim}
\begin{alltt}
\hlcom{# confidence interval}
\hlstd{temp}\hlopt{$}\hlstd{conf.int}
\end{alltt}
\begin{verbatim}
## [1] 0.7321 0.8005
## attr(,"conf.level")
## [1] 0.9
\end{verbatim}
\begin{alltt}
\hlcom{# mean}
\hlstd{temp}\hlopt{$}\hlstd{estimate}
\end{alltt}
\begin{verbatim}
## mean of x 
##    0.7663
\end{verbatim}
\begin{alltt}
\hlcom{# degrees of freedom used in calculating the pvalue or the confidence interval}
\hlstd{temp}\hlopt{$}\hlstd{parameter}
\end{alltt}
\begin{verbatim}
## df 
## 18
\end{verbatim}
\begin{alltt}
\hlcom{# or everything at once}
\hlstd{temp}
\end{alltt}
\begin{verbatim}
## 
## 	One Sample t-test
## 
## data:  ratio
## t = -1.709, df = 18, p-value = 0.1046
## alternative hypothesis: true mean is not equal to 0.8
## 90 percent confidence interval:
##  0.7321 0.8005
## sample estimates:
## mean of x 
##    0.7663
\end{verbatim}
\begin{alltt}
\hlcom{# Answer: A 90% confidence interval for the FEV1/VC ratio is (0.73, 0.80). This includes}
\hlcom{# the normal value of 0.80.}


\hlcom{# However the research question is whether the ratio is lower than normal. This is a one}
\hlcom{# sided hypothesis test, namely the alternative is HA: mu < 0.80.}
\hlkwd{t.test}\hlstd{(ratio,} \hlkwc{mu} \hlstd{=} \hlnum{0.8}\hlstd{,} \hlkwc{alternative} \hlstd{=} \hlstr{"less"}\hlstd{)}
\end{alltt}
\begin{verbatim}
## 
## 	One Sample t-test
## 
## data:  ratio
## t = -1.709, df = 18, p-value = 0.05231
## alternative hypothesis: true mean is less than 0.8
## 95 percent confidence interval:
##    -Inf 0.8005
## sample estimates:
## mean of x 
##    0.7663
\end{verbatim}
\begin{alltt}
\hlcom{# The pvalue is 0.052. The data provide evidence that exposure to B. subtilis may reduce}
\hlcom{# the FEV1/VC ratio, but are inconclusive at the 5% significance level.}
\end{alltt}
\end{kframe}
\end{knitrout}


The R session information (including the OS info, R version and all
packages used):

\begin{knitrout}
\definecolor{shadecolor}{rgb}{0.969, 0.969, 0.969}\color{fgcolor}\begin{kframe}
\begin{alltt}
\hlkwd{sessionInfo}\hlstd{()}
\end{alltt}
\begin{verbatim}
## R version 3.0.2 (2013-09-25)
## Platform: x86_64-apple-darwin10.8.0 (64-bit)
## 
## locale:
## [1] en_US.UTF-8/en_US.UTF-8/en_US.UTF-8/C/en_US.UTF-8/en_US.UTF-8
## 
## attached base packages:
## [1] stats     graphics  grDevices utils     datasets  methods   base     
## 
## other attached packages:
## [1] knitr_1.5
## 
## loaded via a namespace (and not attached):
## [1] evaluate_0.5.1 formatR_0.10   highr_0.3      stringr_0.6.2  tools_3.0.2
\end{verbatim}
\begin{alltt}
\hlkwd{Sys.time}\hlstd{()}
\end{alltt}
\begin{verbatim}
## [1] "2014-03-06 16:31:48 EST"
\end{verbatim}
\end{kframe}
\end{knitrout}




\end{document}
